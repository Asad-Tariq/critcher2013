% Options for packages loaded elsewhere
\PassOptionsToPackage{unicode}{hyperref}
\PassOptionsToPackage{hyphens}{url}
\PassOptionsToPackage{dvipsnames,svgnames,x11names}{xcolor}
%
\documentclass[
  letterpaper,
  DIV=11,
  numbers=noendperiod]{scrartcl}

\usepackage{amsmath,amssymb}
\usepackage{iftex}
\ifPDFTeX
  \usepackage[T1]{fontenc}
  \usepackage[utf8]{inputenc}
  \usepackage{textcomp} % provide euro and other symbols
\else % if luatex or xetex
  \usepackage{unicode-math}
  \defaultfontfeatures{Scale=MatchLowercase}
  \defaultfontfeatures[\rmfamily]{Ligatures=TeX,Scale=1}
\fi
\usepackage{lmodern}
\ifPDFTeX\else  
    % xetex/luatex font selection
\fi
% Use upquote if available, for straight quotes in verbatim environments
\IfFileExists{upquote.sty}{\usepackage{upquote}}{}
\IfFileExists{microtype.sty}{% use microtype if available
  \usepackage[]{microtype}
  \UseMicrotypeSet[protrusion]{basicmath} % disable protrusion for tt fonts
}{}
\makeatletter
\@ifundefined{KOMAClassName}{% if non-KOMA class
  \IfFileExists{parskip.sty}{%
    \usepackage{parskip}
  }{% else
    \setlength{\parindent}{0pt}
    \setlength{\parskip}{6pt plus 2pt minus 1pt}}
}{% if KOMA class
  \KOMAoptions{parskip=half}}
\makeatother
\usepackage{xcolor}
\setlength{\emergencystretch}{3em} % prevent overfull lines
\setcounter{secnumdepth}{-\maxdimen} % remove section numbering
% Make \paragraph and \subparagraph free-standing
\makeatletter
\ifx\paragraph\undefined\else
  \let\oldparagraph\paragraph
  \renewcommand{\paragraph}{
    \@ifstar
      \xxxParagraphStar
      \xxxParagraphNoStar
  }
  \newcommand{\xxxParagraphStar}[1]{\oldparagraph*{#1}\mbox{}}
  \newcommand{\xxxParagraphNoStar}[1]{\oldparagraph{#1}\mbox{}}
\fi
\ifx\subparagraph\undefined\else
  \let\oldsubparagraph\subparagraph
  \renewcommand{\subparagraph}{
    \@ifstar
      \xxxSubParagraphStar
      \xxxSubParagraphNoStar
  }
  \newcommand{\xxxSubParagraphStar}[1]{\oldsubparagraph*{#1}\mbox{}}
  \newcommand{\xxxSubParagraphNoStar}[1]{\oldsubparagraph{#1}\mbox{}}
\fi
\makeatother


\providecommand{\tightlist}{%
  \setlength{\itemsep}{0pt}\setlength{\parskip}{0pt}}\usepackage{longtable,booktabs,array}
\usepackage{calc} % for calculating minipage widths
% Correct order of tables after \paragraph or \subparagraph
\usepackage{etoolbox}
\makeatletter
\patchcmd\longtable{\par}{\if@noskipsec\mbox{}\fi\par}{}{}
\makeatother
% Allow footnotes in longtable head/foot
\IfFileExists{footnotehyper.sty}{\usepackage{footnotehyper}}{\usepackage{footnote}}
\makesavenoteenv{longtable}
\usepackage{graphicx}
\makeatletter
\def\maxwidth{\ifdim\Gin@nat@width>\linewidth\linewidth\else\Gin@nat@width\fi}
\def\maxheight{\ifdim\Gin@nat@height>\textheight\textheight\else\Gin@nat@height\fi}
\makeatother
% Scale images if necessary, so that they will not overflow the page
% margins by default, and it is still possible to overwrite the defaults
% using explicit options in \includegraphics[width, height, ...]{}
\setkeys{Gin}{width=\maxwidth,height=\maxheight,keepaspectratio}
% Set default figure placement to htbp
\makeatletter
\def\fps@figure{htbp}
\makeatother

\KOMAoption{captions}{tableheading}
\makeatletter
\@ifpackageloaded{caption}{}{\usepackage{caption}}
\AtBeginDocument{%
\ifdefined\contentsname
  \renewcommand*\contentsname{Table of contents}
\else
  \newcommand\contentsname{Table of contents}
\fi
\ifdefined\listfigurename
  \renewcommand*\listfigurename{List of Figures}
\else
  \newcommand\listfigurename{List of Figures}
\fi
\ifdefined\listtablename
  \renewcommand*\listtablename{List of Tables}
\else
  \newcommand\listtablename{List of Tables}
\fi
\ifdefined\figurename
  \renewcommand*\figurename{Figure}
\else
  \newcommand\figurename{Figure}
\fi
\ifdefined\tablename
  \renewcommand*\tablename{Table}
\else
  \newcommand\tablename{Table}
\fi
}
\@ifpackageloaded{float}{}{\usepackage{float}}
\floatstyle{ruled}
\@ifundefined{c@chapter}{\newfloat{codelisting}{h}{lop}}{\newfloat{codelisting}{h}{lop}[chapter]}
\floatname{codelisting}{Listing}
\newcommand*\listoflistings{\listof{codelisting}{List of Listings}}
\makeatother
\makeatletter
\makeatother
\makeatletter
\@ifpackageloaded{caption}{}{\usepackage{caption}}
\@ifpackageloaded{subcaption}{}{\usepackage{subcaption}}
\makeatother

\ifLuaTeX
  \usepackage{selnolig}  % disable illegal ligatures
\fi
\usepackage{bookmark}

\IfFileExists{xurl.sty}{\usepackage{xurl}}{} % add URL line breaks if available
\urlstyle{same} % disable monospaced font for URLs
\hypersetup{
  pdftitle={Replication of Study ``How Quick Decisions Illuminate Moral Character'' by Critcher et al.~(2013, Social Psychological and Personality Science)},
  pdfauthor={Asad Tariq (astariq@ucsd.edu)},
  colorlinks=true,
  linkcolor={blue},
  filecolor={Maroon},
  citecolor={Blue},
  urlcolor={Blue},
  pdfcreator={LaTeX via pandoc}}


\title{Replication of Study ``How Quick Decisions Illuminate Moral
Character'' by Critcher et al.~(2013, Social Psychological and
Personality Science)}
\author{Asad Tariq (astariq@ucsd.edu)}
\date{2024-10-12}

\begin{document}
\maketitle

\renewcommand*\contentsname{Table of contents}
{
\hypersetup{linkcolor=}
\setcounter{tocdepth}{3}
\tableofcontents
}

\subsection{Introduction}\label{introduction}

In order to replicate experiment 1 of this study, we will need a sample
size of around 100-120 participants who will read up on a hypothetical
scenario explaining to them two accompanying actions/outcomes: the first
being a moral one and the other being an immoral one. The two actors
involved in the scenario will differ by nature - one of them will be
``quick'' to act, while the other will be ``slow''. The participants
will read the given scenario and answer some questions immediately
afterwards. The questions will revolve around 4 different categories,
namely: quickness, moral character evaluation, certainty and emotional
impulsivity, which the participants will answer on a scale from 1-7.

A potential challenge will be to gather a large enough sample size of
participants similar to the original study (the original study had 119
participants in experiment 1) which are either undergraduates or from a
similar community to conduct the replication of the experiment. Another
challenge would be to establish a reliable way to measure the quickness
of judgement and the moral character of the actors involved in the
scenario, such that the participants are consistent with their responses
across different conditions. Yet another challenge would be the way in
which to convey the notion of ``quick'' and ``slow'' as is intended in
this particular study so that there is no consequential bias in the
evaluations of assessed morality of the two actors.

\href{https://github.com/Asad-Tariq/critcher2013}{GitHub Repository}
\href{https://github.com/Asad-Tariq/critcher2013/blob/main/original_paper/Critcher_et_al_2013.pdf}{Link
to original paper}

\subsection{Methods}\label{methods}

\subsubsection{Power Analysis}\label{power-analysis}

Original effect size, power analysis for samples to achieve 80\%, 90\%,
95\% power to detect that effect size. Considerations of feasibility for
selecting planned sample size.

\subsubsection{Planned Sample}\label{planned-sample}

Planned sample size and/or termination rule, sampling frame, known
demographics if any, preselection rules if any.

\subsubsection{Materials}\label{materials}

All materials - can quote directly from original article - just put the
text in quotations and note that this was followed precisely. Or, quote
directly and just point out exceptions to what was described in the
original article.

\subsubsection{Procedure}\label{procedure}

Can quote directly from original article - just put the text in
quotations and note that this was followed precisely. Or, quote directly
and just point out exceptions to what was described in the original
article.

\subsubsection{Analysis Plan}\label{analysis-plan}

Can also quote directly, though it is less often spelled out effectively
for an analysis strategy section. The key is to report an analysis
strategy that is as close to the original - data cleaning rules, data
exclusion rules, covariates, etc. - as possible.

\textbf{Clarify key analysis of interest here} You can also pre-specify
additional analyses you plan to do.

\subsubsection{Differences from Original
Study}\label{differences-from-original-study}

Explicitly describe known differences in sample, setting, procedure, and
analysis plan from original study. The goal, of course, is to minimize
those differences, but differences will inevitably occur. Also, note
whether such differences are anticipated to make a difference based on
claims in the original article or subsequent published research on the
conditions for obtaining the effect.

\subsubsection{Methods Addendum (Post Data
Collection)}\label{methods-addendum-post-data-collection}

You can comment this section out prior to final report with data
collection.

\paragraph{Actual Sample}\label{actual-sample}

Sample size, demographics, data exclusions based on rules spelled out in
analysis plan

\paragraph{Differences from pre-data collection methods
plan}\label{differences-from-pre-data-collection-methods-plan}

Any differences from what was described as the original plan, or
``none''.

\subsection{Results}\label{results}

\subsubsection{Data preparation}\label{data-preparation}

Data preparation following the analysis plan.

\subsubsection{Confirmatory analysis}\label{confirmatory-analysis}

The analyses as specified in the analysis plan.

\emph{Side-by-side graph with original graph is ideal here}

\subsubsection{Exploratory analyses}\label{exploratory-analyses}

Any follow-up analyses desired (not required).

\subsection{Discussion}\label{discussion}

\subsubsection{Summary of Replication
Attempt}\label{summary-of-replication-attempt}

Open the discussion section with a paragraph summarizing the primary
result from the confirmatory analysis and the assessment of whether it
replicated, partially replicated, or failed to replicate the original
result.

\subsubsection{Commentary}\label{commentary}

Add open-ended commentary (if any) reflecting (a) insights from
follow-up exploratory analysis, (b) assessment of the meaning of the
replication (or not) - e.g., for a failure to replicate, are the
differences between original and present study ones that definitely,
plausibly, or are unlikely to have been moderators of the result, and
(c) discussion of any objections or challenges raised by the current and
original authors about the replication attempt. None of these need to be
long.




\end{document}
